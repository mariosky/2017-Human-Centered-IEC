
\documentclass{llncs}
\usepackage{graphicx}        % standard LaTeX graphics tool
                             % when including figure files
\usepackage{url}
%%%%%%%%%%%%%%%%%%%%%%%%%%%%%%%%%%%%%%%%%%%%%%%%%%%%%%%%%%%%%%%%%%%%%%%%%%%%%%%%%%%%%%%%%

\begin{document}
\sloppy

\title{Increasing User Engagement in a Volunteer-Based Ephemeral Evolutionary Computation System}
\titlerunning{Increasing Engagement in a Volunteer-Based Ephemeral Evolutionary Computation System}


\author{Mario Garc\'ia-Valdez\inst{1} \and Juan J. Merelo Guerv\'os\inst{2} \and  Lucero Lara \inst{1}}

\institute{Instituto Tecnol\'ogico de Tijuana, Tijuana BC, Mexico
\and
Universidad de Granada, Granada, Spain
\email{mario@tectijuana.edu.mx}\\
\email{jmerelo@geneura.ugr.es}}

\authorrunning{Garc\'ia-Valdez, Merelo, \& Lara }

\maketitle


\begin{abstract}

Volunteer computing is a distributed computing system, in which people
provide their own computing resources or storage to contribute to a common effort.
By runnning a script in a web page, collaboration is straightforward, but also ephemeral.
Resources depend on the amount of time a user lends, whicn means that 
the user has to be kept engaged to obtain as many computing cycles as
possible.

In this paper, we analyze a volunteer-based evolutionary computing system called
NodIO with the objective of discovering rules that encourage volunteer
participation, making it more or less efficient. We present the results of
an experiment where a gammification technique is applied by adding a leaderboard 
showing the top registered contributors. In NodIO volunteers can
participate without the need to create an account, so the question was
if the need to register would hava a negative impact on user participation. 
The experiment results show that even if only a small percentege of users created an account,
those participating in the competition provided around 90.

\keywords{Distributed Evolutionary Algorithms, Volunteer Computing}
\end{abstract}



  
\end{document}
