\section{Introduction}
% Human Based Computation
% C-IEAs
% Volunteers 
% Problems
% Human Centered
  % User -> Human
  % Social Network
  % Devices IoT
  % Activities
  % Engagement
% C-IEA Complex Interactions
% Why an HC Framework (Objectives)  
% Presentation 
Interactive evolutionary computation (IEC) systems are, in general, methods
of evolutionary computation whose fitness evaluations are performed by humans 
within an interactive system \cite{eiben2015interactive}.  Human evaluation 
is needed because the fitness function is not known and the result
of optimization should fit certain human need. Usually human preference is
needed to evaluate subjective characteristics like visual appeal or 
attractiveness \cite{biomorphs}, but in other cases human behavior is 
considered, for instance to optimize teamwork \cite{kosorukoff2002evolutionary}
or creativity cite[cook]. When human interaction is responsible of other 
operations of the evolutionary process some authors classify these IEC methods 
as human-based evolutionary computation cite[kosorukoff] or as human-based computation cite[].
IEC systems have demonstrated their ability for effectively
producing art and design \cite{Bentley:1999:intro,Sims:1991,todd:1992,evoeco},
as well as other artifacts in many other domains \cite{ie1}. 
In order to reach more human users, Interactive Evolutionary Algorithms (IEAs) are
some times developed as web applications,
who depend on volunteers who visit the system to interactively help
with the search, using both anonymous and registered users. Some systems 
employ a collaborative technique, where several users assign a quality assessment 
to a single solution and then an aggregated fitness value is calculated 
\cite{picbreeder,seyama2016development,wagy2014collective}.
However, the necessary intervention of humans in these systems leads
to them having several inherent problems arising from the very nature of 
the algorithms, namely, human evaluations are slow and expensive, there is a
human fatigue caused by the interaction \cite{ie1}, and
the boredom arising when users evaluate a large number of phenotypes 
many of which are not interesting or are very similar to each other. 
The number of evaluations that IEC can receive is limited by the number of users
collaborating and the amount of interaction of each. Having a volunteer system 
can lower the requirements of anyone participating in
the experiment thus increasing the {\em performance}, in terms of {\em
human brain cycles}, of the whole system. Using a volunteer based system  
brings other issues \cite{sarmenta2001volunteer,web:BOINC}, 
such as the volunteer\'s lack of accountability,
and the need to build trust between participants and application providers. 
For the provider there is also the difficulty of establishing 
the amount of time and resources
a volunteer is willing to spend on the system, and how they decide if they
participate or not \cite{JJ:2016}. 

In order to increase volunteer participation and to tackle some of the issues mentioned above,  
IEAs must developed following a human centered design \cite{greenhouse2012human},
giving extensive attention to volunteers, not only because their
explicit evaluation is essential, but also because the context of the 
interaction affects the system as a whole. In collaborative systems, users interact 
with each other, and are aware of the actions. Users also consider their previous evaluations
and past experiences.

%Please help with this section, 
Evolutionary computations algorithms are 


In this work we propose giving the same
importance to users and their interactions as the population of 
individuals have in a traditional EA, even to the extent of using this
knowledge as an integral part of the evolutionary algorithm.
Instead of letting the user assign fitness directly via a rating
or selection system, fitness assignment could depend on the actions of
a group of users, possibly connected in a social network, and depend on how 
they tag, share, rate, or even delete the phenotype. The selection of parents could
also depend on the previous actions, leveraging information such as the fact that  they have the same tag, or shared by
similar users, or whether a leader of opinion liked both or if they are stored in the same collection.
That knowledge could also be applied in choosing which phenotypes would be presented to a
user, in order to give them those they will find more interesting or useful. 




\section{Related Work}
% C-IEAS
% Volunteer Based
% HBC
% Engagement Techniques
% Models

\section{HC  IEC Framework}
  \subsection{Real World}
  \subsection{Data Model}
  \subsection{Evolutionary Algorithm}

\section{EvoSpace-Interactive}
  \subsection{Interfaces}
  \subsection{Data Models}
    \subsubsection{EvoSpace}
    \subsubsection{Graph}
    \subsubsection{PostgreSQL}
  \subsection{Evolution}

\section{Case Studies}
\subsection{EvoDraw}
\subsection{EvoDraw-Kinect}
\subsection{XYZ-Unplugged}

\section{Discussion}
\section{Conclusions}

\begin{acks}
  The authors would like to thank Dr. Yuhua Li for providing the
  matlab code of  the \textit{BEPS} method. 

  The authors would also like to thank the anonymous referees for
  their valuable comments and helpful suggestions. The work is
  supported by the \grantsponsor{GS501100001809}{National Natural
    Science Foundation of
    China}{http://dx.doi.org/10.13039/501100001809} under Grant
  No.:~\grantnum{GS501100001809}{61273304}
  and~\grantnum[http://www.nnsf.cn/youngscientsts]{GS501100001809}{Young
    Scientsts' Support Program}.

\end{acks}
