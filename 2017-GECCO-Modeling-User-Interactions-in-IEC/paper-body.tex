\section{Introduction}

The intervention of humans brings particular challenges
to designers of Interactive evolutionary computation (IEC) systems.
Namely, human evaluations are scarce, slow and expensive, there is human 
fatigue caused by the interaction \cite{ie1}, and
also boredom arises when users evaluate a large number of phenotypes.
The general goal of this research is to develop a  human-centered 
\cite{gasson2003human} software framework that can be used to 
increase volunteer participation in C-IEC systems by using gamification techniques.
Gamification is the use of game design elements in non-game contexts \cite{deterding2011game}. 
The gamification element employed in this work is a rewarding mechanism
\cite{dubois2013understanding}. An example of the use of this technique is a recent
work by Wagy \& Bongard \cite{wagy2014collective} where user interaction is
needed for developing new designs of robot locomotion.
Collaboration is encouraged by gamifying the system using the maximum distance
indicator to inspire the user to try and ``beat'' previous designs. Also Seyama and
Munetomo \cite{seyama2016development} propose the reduction of user
fatigue by using a collaborative filtering algorithm to show only the
information utilized by similar users. In Section \ref{sec:framework}
we present our proposed framework. Next the EvoDrawings application case study 
is presented in Section \ref{sec:evodraw} and the
results are discussed in Section \ref{sec:results}. Finally, some concluding
remarks are provided in Section \ref{sec:conclusions}.

\section{Human-Centered C-IEC Framework}
\label{sec:framework} 
The main design considerations of the framework are explained next: 
\begin{itemize}
\item {\bf Users are volunteers.} Users donate their computing resources, so they are 
unaccountable and sometimes they try to game the system. Project owners must actively promote and
design the interactive system to engage volunteers \cite{oh2015clicking}. % Define
\item {\bf Users are not alone}
  Relationships between users in an interactive evolutionary algorithm can be modeled
  as a social network, with well established semantics, algorithms and metrics 
  \cite{ahuja1993network}. 
\item {\bf Interaction is a stream of actions.}
  Real time processing of users' actions could be needed for certain applications when data is 
  captured by sensors, or directly captured as user input. For example, social networks encourage
  users to publish their interactions with other users, media objects and places.
  there are iniciatives like the W3C Activity Streams 2.0  specification, used for 
  representing common activities in social web applications \cite{json:streams}. 
\end{itemize}

A graph is proposed for modeling the social network of users and their interactions 
with candidate solutions, and the relationships between them in the population.
The graph database system used in the implementation is Neo4J, which is a scalable solution 
\cite{miller2013graph,holzschuher2013performance}. 
This graph is also used to increase engagement through gamification, as expleained in the next section.


\section{Case Study: EvoDrawings Gamification}
\label{sec:evodraw}
As a case study, a C-IEC application was developed by extending the 
EvoSpace-Interactive (ES-I) platform \cite{garcia2013evospace}. 
The open source code is hosted in GitHub at
\url{https://github.com/mariosky/evo-drawings}. The rewarding 
mechanism as it is applied in EvoDrawings gives more importance 
to the preference of those users with higher reputation
as given by their score points and experience levels.  
Each time a user does on of these actions their score is incremented by one:
start a session, rate a phenotype, create a collection, save a phenotype of 
the wall to a collection, save a phenotypes from a friend's collection, and
explore collections of other friends. Each of these actions are stored in
the Neo4J graph. Two variables are used to determine the weight of a user's 
preference:

\begin{itemize}
\item {\bf Experience}: This variable depends on the score and is a value 
between 0 and 100. A new user starts at zero, and the experience increases until
it reaches 100 actions.

\item {\bf Participation}: This variable is simply the degree of the user node 
in the graph (number of edges).    
\end{itemize}

Three versions of EvoDrawings were compared: Base (B): All users have the same weight,
Non Graph Gamification (G): Only experience is consideredm and Graph Gamification (GG): 
Both experience and participation are considered.  

\section{Results}
\label{sec:results} 
Before release, the deployment was first tried with a few beta testers. 
When applying the leader board gamification technique for the first time a 
problem was found: some users were cheating by giving a
rating to an animation even before it was returned from the server, this was done by just
constantly clicking the mouse button. This is a common problem found in systems using leader
boards because by making the scores visible to other players they are encouraged 
to compete \cite{hickman2010total}. The version used in experiments disabled the button until 
the drawing animation was over. The results of each of the three experiments in 
terms of participation are detailed next.

\begin{table}
  \small
  \caption{After a week of the announcement the total number of volunteers, 
  nodes and edges in the graph and analytics URLs}
  \label{tab:urls} 
  \centering
  \small
  \begin{tabular}{l l l l l}
    \hline\noalign{\smallskip}
     Deployment &  Users &  Nodes &  Edges \\
    \noalign{\smallskip}\hline\noalign{\smallskip}
    B   & 53 &  595   & 2220   \\ \hline
    G   & 54 &  648   & 2596   \\ \hline
    GG  & 68 &  932   & 3594   \\ \hline
    \end{tabular}
\end{table}

Table \ref{tab:urls} shows the total number of volunteers, nodes and edges 
in the graph after each experiment. Moreover, the total number of evaluated 
phenotypes for each volunteer is presented in figure 
\ref{fig:top-ranked-participation} where users are ranked by the 
number of phenotypes they rated. When comparing all the experiments the deployment GG had the higher
number of participation, besides attracting also the higher number of users.    

  
\section{Conclusions}
\label{sec:conclusions}

In concordance with the results obtained in other 
browser-based volunteer systems, after 
applying the gamification techniques, user participation was increased. 

One of the interesting future lines of work would be to look a bit
more closely at the behavior of users as they are rating artifacts 
in the web system. These initial experiments hint at a possible power law, which might indicate that
the IEC system could be self-organizing, a process that would allow it
to reach a critical state, as has been found in software repositories,
for instance \cite{Merelo2016:repomining}. 

One of the interesting future lines of work would be to study the possible negative effects of using  
gamification techniques to improve engagement, like cheating or
literally {\em gaming} the system to defeat competition. 
Finally, the refinement of the proposed Human-Centered framework will need
more case studies and further multi-disciplinary research. 

\begin{acks}
This work has been supported in part by: de Ministerio espa\~{n}ol de Econom\'{\i}a y Competitividad under project TIN2014-56494-C4-3-P (UGR-EPHEMECH) and by Consejo Nacional de Ciencia y Tecnolog\'{\i}a under project 220590.




\end{acks}
