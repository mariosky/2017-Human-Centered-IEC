\section{Introduction}
% Main Ideas:

% IEAs
% Human Based Computation
% C-IEAs
% Problems
% Volunteers 
% Human Centered
% C-IEA Complex Interactions
  % User -> Human
  % Social Network
  % Devices IoT
  % Activities
  % Engagement

% Why an HC Framework (Objectives)  
% Presentation 


% IEAs
Interactive evolutionary computation (IEC) systems are, in general, evolutionary methods
whose fitness evaluations are performed by humans through an
interactive system \cite{eiben2015interactive}. They are usually
applied in problems where the fitness function is not known  or simply
does not exist, and the result
of optimization should fit certain human and possibly aesthetic
need. That is why their use case includes the evolution of objects
with subjective characteristics such visual appeal or 
attractiveness \cite{biomorphs} as well as others where human behavior is 
considered, for instance to optimize teamwork \cite{kosorukoff2002evolutionary}
or creativity cite[cook].
% Human Based Computation
When human interaction is responsible of other 
operations of the evolutionary process some authors classify these IEC methods 
as human-based evolutionary computation cite[kosorukoff] or as human-based computation cite[].
IEC systems have demonstrated their ability for effectively
producing art and design \cite{Bentley:1999:intro,Sims:1991,todd:1992,evoeco},
as well as other artifacts in many other domains \cite{ie1}. 
% C-IEAs
In order to reach more human users, Interactive Evolutionary Algorithms (IEAs) are
some times developed as web applications,
who depend on volunteers who visit the system to interactively help
with the search, using both anonymous and registered users. Some systems 
employ a collaborative technique, where several users assign a quality assessment 
to a single solution and then an aggregated fitness value is calculated 
\cite{picbreeder,seyama2016development,wagy2014collective}.
% Problems
However, the necessary intervention of humans in these systems leads
to them having several inherent problems arising from the very nature of 
the algorithms, namely, human evaluations are slow and expensive, there is a
human fatigue caused by the interaction \cite{ie1}, and
the boredom arising when users evaluate a large number of phenotypes 
many of which are not interesting or are very similar to each other. 
The number of evaluations that IEC can receive is limited by the number of users
collaborating and the amount of interaction of each. Having a volunteer system 
can lower the requirements of anyone participating in
the experiment thus increasing the {\em performance}, in terms of {\em
human brain cycles}, of the whole system.
% Volunteers 
Using a volunteer based system raises other issues \cite{sarmenta2001volunteer,web:BOINC}, 
such as the volunteer\'s lack of accountability,
and the need to build trust between participants and application providers. 
For the provider there is also the difficulty of establishing 
the amount of time and resources
a volunteer is willing to spend on the system, and how they decide if they
participate or not \cite{JJ:2016}. 
% Human Centered
In order to increase volunteer participation and to tackle some of the issues mentioned above,  
we proposed a framework following a human centered design \cite{greenhouse2012human},
giving extensive attention to volunteers, not only because their
explicit evaluation is essential, but also because the context of the 
interaction affects the system as a whole. In collaborative systems, users interact 
with each other, and are aware of the actions ref[]. Users also consider their previous evaluations
and past experiences ref[].
%Please help with this paragraph, is evidence needed?
% It's the tuning of the algorithm which is data driven, not the
% algorithm itself, right? - JJ
In evolutionary computation algorithms are data driven since it is important to keep data about the 
population of each generation, in order to generate the next or even change certain parameters
or analyze the evolutionary process. 
% Human Centered Importance
The framework models the context of interaction along with the population and
this knowledge is available to be an integral part of the evolutionary computation.
%This could be generalized?
For example, in a C-IEC application, fitness assignment depend on the
actions of a social network of users. 
A stream of actions is triggered when they tag, share, rate, store or even delete a phenotype. The selection of parents could
also depend also on the previous actions, leveraging information such as the fact that they have the same tag, or are shared by
similar users. That knowledge could also be applied in choosing which phenotypes present to a certain
users, in order to give them those they will find more interesting or useful. 
% IoT
The context of interaction, can also include data retrieved from sensors and other IoT devices.
% Engagement
Data available from the interaction is also used to increase the performance of the system by applying 
gamification techniques. Gamification is defined by Deterding et al. as
``the use of game design elements in non-game contexts'' \cite{deterding2011game}.
The gamification element employed in a case study is a rewarding mechanism  
\cite{dubois2013understanding}, in general they consist of a reputation system with score points, 
levels and leader boards. Points are awarded to users in response of 
the accomplishment of certain activities that need to be encouraged. Levels depend
on the score and certain features of the game are only available to gamers when 
they reach a giving level.

Three case studies are presented as a proof of concept, providing both
conceptual and implementation details of the framework. 
The experiments were implemented using the EvoSpace-Interactive framework \cite{garcia2013evospace}, 
and where employed in different contexts of interaction. 
Additional details of each case study application is presented elsewhere refs[][][]. 
Giving this paper the objective of presenting the underlying conceptual framework.     

The rest of this paper is organized as follows.
Section \ref{sec:interactive} presents related work on the topic 
of Interactive Evolution.
Then, Section \ref{sec:e} presents the Model of Human Interactions in Collaborative Interactive 
 Evolutionary Computation which 
the main proposal of this work. An implementation using the EvoSpace-Interactive  framework is detailed in Section \ref{sec:HCF}.
The case studies  and results are presented in Section \ref{sec:experiments}.
Finally, a concluding remarks are provided in Section \ref{sec:conclusions}.

\section{Related Work}
% C-IEAS
% Volunteer Based
% HBC
% Engagement Techniques
% Models

\section{HC  IEC Framework}
  \subsection{Real World}
  \subsection{Data Model}
  \subsection{Evolutionary Algorithm}

\section{EvoSpace-Interactive}
  \subsection{Interfaces}
  \subsection{Data Models}
    \subsubsection{EvoSpace}
    \subsubsection{Graph}
    \subsubsection{PostgreSQL}
  \subsection{Evolution}

\section{Case Studies}
\subsection{EvoDraw}
\subsection{EvoDraw-Kinect}
\subsection{XYZ-Unplugged}


\section{Conclusions}

\begin{acks}
  % The authors would like to thank Dr. Yuhua Li for providing the
  % matlab code of  the \textit{BEPS} method. 

  % The authors would also like to thank the anonymous referees for
  % their valuable comments and helpful suggestions. The work is
  % supported by the \grantsponsor{GS501100001809}{National Natural
  %   Science Foundation of
  %   China}{http://dx.doi.org/10.13039/501100001809} under Grant
  % No.:~\grantnum{GS501100001809}{61273304}
  % and~\grantnum[http://www.nnsf.cn/youngscientsts]{GS501100001809}{Young
  %   Scientsts' Support Program}.

\end{acks}
